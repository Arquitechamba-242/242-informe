\documentclass[../main]{subfiles}

\begin{document}
\section{Conclusiones}
    El Sistema de Control de Invernadero Automatizado a desarrollar será
    una solución efectiva para el monitoreo y control de las condiciones ambientales
    dentro de un invernadero.
    El uso del microcontrolador ESP32, en conjunto con sensores de temperatura,
    humedad y luz, ha permitido la automatización del riego, la ventilación y el
    ajuste de otras variables esenciales.
    El sistema demostrará ser capaz de adaptarse a los cambios en el ambiente y de
    mantener los parámetros en niveles óptimos para el crecimiento de las plantas.
    Además, la integración de un sistema de comunicación inalámbrica ha facilitado
    el monitoreo remoto en tiempo real, lo cual es una ventaja significativa para
    los usuarios que deseen controlar el invernadero a distancia.
    Sin embargo, durante el desarrollo del proyecto se identificaron desafíos
    relacionados con la precisión de los sensores y la optimización de la
    comunicación, los cuales representan áreas de mejora para futuras versiones del
    sistema.
\end{document}
