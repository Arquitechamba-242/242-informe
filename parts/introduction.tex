\documentclass[../main]{subfiles}

\begin{document}
\section{Introducción}

El proceso de cultivar plantas en un entorno controlado requiere la observación
y moderación de múltiples condiciones ambientales necesarias para el crecimiento
de las plantas de interés.
El presente proyecto está orientado a facilitar la labor de registrar e
inspeccionar indicadores cuantitativos que corresponden con la situación de un
ambiente de cultivo.

En su centro se encuentra un microcontrolador ESP32 \supercite{devkitv4} que, en
conjunto con sensores colocados en el ambiente a monitorear, constituye un
bloque fundacional en el monitoreo de un sitio de cultivo.
Además, sus capacidades de comunicación inalámbrica permiten integrar la
recolección de datos con servicios en la nube sobre los cuales es posible
construir un sistema de alertas frente a posibles condiciones inadecuadas.

El presente documento describe el trabajo realizado según las ideas mencionadas.
\end{document}
