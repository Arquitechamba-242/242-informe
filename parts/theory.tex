\documentclass[../main]{subfiles}

\begin{document}
\section{Fundamento teórico}

\subsection{Hardware}

\subsubsection{Microcontrolador}

ESP32 es una familia de microcontroladores que pertenece a Espressif
caracterizada por su variedad de módulos integrados orientados a
telecomunicaciones por Bluetooth y Wi-Fi. \supercite{ESP32_espressif}

La serie ESP32-DevKitC, en particular, es notable por tener una abundancia de
pines expuestos para facilitar su conexión y uso.
El presente prototipo utiliza un microcontrolador ESP32-DevKitC V4, mismo que
tiene integrados un botón de reinicio, así como un puerto Micro-USB
conectado a un puente USB-a-UART \supercite{devkitv4}.
Una imagen del microcontrolador con detalles anotados es provista en la figura
\ref{esp32devkitcv4image}.

Entre sus componentes, son de interés un módulo ESP32-WROOM-32 integrado con un
microprocesador ESP-D0WDQ6, una interfaz serial de periféricos con
\qty{4}{\mega\byte} de memoria flash integrada, una antena y un oscilador de
cristal con frecuencia \qty{40}{\MHz}, \supercite{esp32wroom32doc}
tal como se puede observar en la figura \ref{esp32wroom32esq}.

\begin{figure}[H]
	\centering
	\includegraphics[width=0.95\textwidth]{res/esp32-devkitc-v4-functional-overview.jpg}
	\caption{Descripción del ESP32-DevKitC V4 \supercite{devkitv4}}
	\label{esp32devkitcv4image}
\end{figure}

\begin{table}[H]
	\centering
	\begin{tabularx}{0.9\textwidth}{l X}
		\toprule
		\multicolumn{1}{c}{\textbf{ Componente }} &
		\multicolumn{1}{c}{\textbf{ Descripción }}                                                                                                                                                                \\
		\midrule
		ESP32-WROOM-32                            & Módulo con el ESP32.                                                                                                                                          \\
		EN                                        & Botón de reinicio.                                                                                                                                            \\
		Boot                                      & Botón de descarga. Mantenerlo presionado y luego presionar \textit{EN} inicia el \textit{Modo de descarga de firmware}, que opera a través del puerto serial. \\
		USB-to-UART Bridge                        & Un chip de puente USB-UART. Provee velocidades de transferencia de hasta \qty{3}{\mega\byte\per\s}.                                                           \\
		Micro USB Port                            & Interfaz USB. Provee energía a la placa y puede comunicar a una computadora con el módulo ESP32-WROOM-32.                                                     \\
		5V Power On LED                           & Se enciende cuando está conectado a la placa un suministro de energía de \qty{5}{\V} externo o por USB.                                                       \\
		I/O                                       & La mayoría de los pines del módulo ESP están conectados con los que sobresalen de la placa. Pueden poseer funcionalidad I2C, SPI, etc.                        \\
		\bottomrule
	\end{tabularx}
	\caption{ESP32-DevKitC V4 con el módulo ESP32-WROOM-32 \supercite{devkitv4}}
	\label{esp32devkitcv4_descrp}
\end{table}

\begin{landscape}
	\begin{figure}[H]
		\centering
		\includegraphics[height=0.85\textheight]{res/esp32-wroom-32_diagram.jpg}
		\caption{Esquema del ESP32-WROOM-32 \supercite{esp32wroom32doc}}
		\label{esp32wroom32esq}
	\end{figure}
\end{landscape}

\begin{figure}[H]
	\centering
	\includegraphics[width=\textwidth]{res/esp32_datasheet_en_page-0013.jpg}
	\caption{Diagrama de pines del ESP32-D0WDQ6 \supercite{esp32d0wdq62doc}}
	\label{fig:esp32d0wdq62pines}
\end{figure}

\subsubsection{Sensor BME280}

El sensor BOSCH-BME280 es capaz de medir humedad relativa, presión
barométrica y temperatura ambiente. \supercite{boschbme280descr}
Algunas características se encuentran en la tabla \ref{tab:bme280techdata}.

\begin{figure}[H]
	\centering
	\includegraphics[height=6cm]{res/sensor-bme280-presion-temperatura-y-humedad.jpg}
	\caption{Sensor BME280 \supercite{bme280image}}
	\label{fig:bme280fig}
\end{figure}

\begin{table}[H]
	\centering
	\begin{tabularx}{0.9\textwidth}{l X}
		\toprule
		Rango de operación                       & Presión: \qtyrange{300}{1100}{\hecto\Pa}                           \\
		                                         & Temperatura: \qtyrange{-40}{85}{\degreeCelsius}                    \\
		\midrule
		Voltaje de entrada V\textsubscript{DD}   & \qtyrange{1.71}{3.6}{\V}                                           \\
		Voltaje de entrada V\textsubscript{DDIO} & \qtyrange{1.2}{3.6}{\V}                                            \\
		\midrule
		Interfaz                                 & I\textsuperscript{2}C y SPI                                        \\
		\midrule
		Consumo de corriente                     & \qty{1.8}{\micro\A} @ \qty{1}{\Hz} (humedad, temperatura)          \\
		                                         & \qty{2.8}{\micro\A} @ \qty{1}{\Hz} (presión, temperatura)          \\
		                                         & \qty{3.6}{\micro\A} @ \qty{1}{\Hz} (humedad, presión, temperatura) \\
		                                         & \qty{0.1}{\micro\A} (inactivo)                                     \\
		\midrule
		\textbf{Sensor de humedad}               &                                                                    \\
		Tiempo de respuesta                      & \qty{1}{\s}                                                        \\
		Tolerancia de precisión                  & \qty[parse-numbers=false]{\pm\ 3}{\percent} humedad relativa       \\
		Histéresis                               & \qty[parse-numbers=false]{\leq 2}{\percent} humedad relativa       \\
		\midrule
		\textbf{Sensor de presión}               &                                                                    \\
		Ruido RMS                                & \qty{0.2}{\Pa}                                                     \\
		Error de sensitividad                    & \qty[parse-numbers=false]{\pm\ 0.25}{\percent}                     \\
		Coeficiente de temperatura ajustado      & \qty[parse-numbers=false,per-mode=fraction]{\pm\ 1.5}{\Pa\per\K}   \\
		\midrule
		Base                                     & LGA de 8 pines con tapa de metal                                   \\
		Dimensiones                              & \qtyproduct{ 2.5 x 2.5 x 0.93 }{\mm\cubed}                         \\
		\bottomrule
	\end{tabularx}
	\caption{Información técnica del BME280 \supercite{boschbme280techdata}}
	\label{tab:bme280techdata}
\end{table}

\subsubsection{Sensor MQ135}

El sensor MQ135 mide la calidad del aire utilizando \ce{SnO2}, cuya
conductividad varía cuando entra en contacto con impurezas en el aire.
\supercite{mq135winson}
En la tabla \ref{tab:mq135conditions} son mencionadas algunas características
del sensor.

\begin{figure}[H]
	\centering
	\includegraphics[scale=0.5]{res/sensor-mq-135-gas-calidad-aire.jpg}
	\caption{Sensor MQ-135 \supercite{mq135image}}
	\label{fig:mq135fig}
\end{figure}

\begin{table}[H]
	\centering
	\begin{tabular}{m{6cm} m{6cm}}
		\toprule
		Parámetro                                      & Condición                                                                                      \\
		\midrule
		\textbf{Condiciones de trabajo}                &                                                                                                \\
		Voltaje en circuito                            & \qty{5.0(1)}{\V} AC o DC                                                                       \\
		Voltaje de calentamiento                       & \qty{5.0(1)}{\V} AC o DC                                                                       \\
		Resistencia de calentador                      & 33 $\Omega\ \pm$ 5\%                                                                           \\
		\midrule
		\textbf{Condiciones de ambiente}               &                                                                                                \\
		Temperatura de uso                             & \qtyrange{-10}{45}{\degreeCelsius}                                                             \\
		Humedad relativa permitida                     & \qty[parse-numbers=false]{< 95}{\percent}                                                      \\
		Valores de concentración de oxígeno permitidos & \qty{21}{\percent} en condiciones normales, mínimo de \qty[parse-numbers=false]{> 2}{\percent} \\
		\bottomrule
	\end{tabular}
	\caption{Condiciones de trabajo normales y de ambiente permitidos del MQ-135 \supercite{mq135hanwei}}
	\label{tab:mq135conditions}
\end{table}

\subsubsection{Sensor ECH2O EC-5}

El sensor EC-5 mide la humedad del suelo mediante la resistencia entre dos
electrodos, la cual varía con proporcionalidad inversa respecto a la humedad
del suelo cuando el sensor está inserto en él.
\supercite{ec5humedadsuelosensor}
Algunas especificaciones pueden encontrarse en la tabla \ref{tab:ec5tecesp}

\begin{figure}[H]
	\centering
	\includegraphics[scale = 0.6]{res/sonde-ec-5-decagon_.png}
	\caption{Sensor Decagon EC-5 \supercite{ec5image}}
	\label{fig:ec5fig}
\end{figure}

\begin{table}[H]
	\centering
	\begin{tabular}{c c}
		\toprule
		\multicolumn{1}{c}{\textbf{ Parámetro }} &
		\multicolumn{1}{c}{\textbf{ Valor }}                                                                 \\
		\midrule
		Tiempo de medición                       & \qty{10}{\ms}                                             \\
		Precisión                                & \qty[parse-numbers=false]{\pm 2}{\percent}                \\
		Requisitos de energía                    & \qtyrange{2.5}{3.6}{\V\textsubscript{DC}} @ \qty{10}{\mA} \\
		Rango de temperaturas                    & \qtyrange{-40}{60}{\degreeCelsius}                        \\
		Rango de medición                        & De 0 hasta saturarse                                      \\
		\bottomrule
	\end{tabular}
	\caption{Especificaciones técnicas del ECH2O EC-5 \supercite{ec5humedadsuelosensor}}
	\label{tab:ec5tecesp}
\end{table}

\end{document}
